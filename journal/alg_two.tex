\section{The Defensive Player}\label{method2}

The optimal control problem introduced in~\eqref{costFnc1} can be solved directly without the additional Cartesian constraints considered in the previous section. As opposed to fixing a desired landing point and a desired landing time to satify the requirements of a higher-level strategy, there can be times during table tennis where it is much more important to safely return the ball. A \emph{defensive} table tennis player could relax the previously imposed racket constraints $\mbox{\eqref{transCond1} -- \eqref{transCond3}}$ by requiring only that the task constraints $\mbox{\eqref{hitFunc} -- \eqref{landFunc}}$ are satisfied.
%
\subsection{Table Tennis Task Constraints}
%
The indoor environment that is modeled contains a standard ping pong table with coordinates 
%
\begin{equation}
\begin{aligned}
\tabletennis = \{(x,y, \ztable) \in \mathbb{R}^{3} | & \ \scalebox{0.75}[1.0]{\( - \)}\tfrac{w_{T}}{2} \leq x \leq \tfrac{w_{T}}{2}, \\
& \ y_{\mathrm{edge}} - l_{T} \leq y \leq y_{\mathrm{edge}} \},
\end{aligned}
\end{equation}
%
\noindent where the origin is placed at the robot base. The table with width $w_T = 152$ cm and length $l_T = 276$ cm is approximately at $z_{T} = -0.89$ cm height and placed $|y_{\mathrm{edge}}| = 115$ cm away from the robot base, see Figure~\ref{robot}. The racket and the table tennis ball have a radius of $\racketRadius \approx 7.6 $ cm and $\ballRadius = 2$ cm, respectively. The condition for successful landing can be put succinctly as follows: the ball after the hit has to pass over the net, below the wall and land on the opponents court. See Figure~\ref{tableTennisConstraints} for an illustration. %\footnote{This definition eliminates all illegal cases, constraints on landing velocities are not necessary.}. 
%
\paragraph{\textbf{Hitting Constraint}.} All possible impacts of the racket with the ball at time $\hitTime$ are captured by the \emph{hitting set} $\hit$
%
%
\begin{align}
\hit = \{(&T,\ \racket(T),\ \normal(T)) \in \mathbb{R}^{7} \ | \ T \geq 0, \\
&\ 0 \leq \normal(T)^{\mathrm{T}}(\ballPred(T) - \racket(T)) \leq \ballRadius, \\
& \|\projRacket(T)(\ballPred(T) - \racket(T))\| \leq \racketRadius \}, \label{hitCond}
\end{align}
%
\noindent where $\projRacket(t) = \vec{I} - \normal(t)\normal^{\mathrm{T}}(t)$ is the projection matrix onto the racket plane. The hitting function
%
\begin{align}
&\hitFun\big(\joint(T), T\big) = \begin{pmatrix}
T \\ \kin_{p}(\joint(T)) \\ \kin_{n}(\joint(T))
\end{pmatrix} \label{switchCond1} 
\end{align}
%
\noindent enforces the kinematic constraints for hitting when $\hitFun\big(\joint(T), T\big) \in \hit$.
%
\paragraph{\textbf{Net Constraint}.} When crossing the net at time $t$, the ball should be above the net height $z_{\mathrm{net}}$ and below the wall $\zwall$, that is, $(t,\ \ballPred(t))$ should belong to the set 
%
\begin{equation}
\begin{aligned}
\net = \{(\netTime, \ &\ballPred(\netTime))\in \ \mathbb{R}^{4} \ | \ \netTime > 0, \\ 
&b_y(\netTime) = y_{\mathrm{net}} := y_{\mathrm{edge}} - \tfrac{l_{T}}{2}, \\
&\znet \leq \ b_z(\netTime) \leq \zwall \}. \label{netCond}
\end{aligned}
\end{equation}
%
The net hitting time $\netTime$ is calculated by using the ball prediction functions,
%
\begin{align}
\netTime(\joint(T),\dot{\joint}(T),T) = \{t \ | \ b_y(t) = y_{\mathrm{net}}\}. \label{netTime}
\end{align}
%
The net function $\netFun$ that predicts the future ball position on the vertical net plane
%
\begin{align}
&\netFun\big(\joint(T),\dot{\joint}(T),T\big) \!= \!\begin{pmatrix}
\netTime \\ b_x(\netTime) \\ y_{\mathrm{net}} \\ b_z(\netTime)
\end{pmatrix}\label{switchCond2}
\end{align}
%
\noindent is then the composition of a ball-racket contact model with the ball flight model.
%
\paragraph{\textbf{Landing Constraint}.} The desired condition for landing afterwards in the opponents court will then be 
%
\begin{equation}
\begin{aligned}
\landEvent = \{(\landTime, \ballPred&(\landTime)) \in \mathbb{R}^{4} \ | \ \landTime > \netTime, \\
& \qquad \quad \ b_z(\landTime) = \ztable + \ballRadius, \\
& \qquad -\tfrac{w_{T}}{2} \leq b_x(\landTime) \leq \tfrac{w_{T}}{2}, \\
&y_{\mathrm{net}} - \tfrac{l_{T}}{2}\leq \ b_y(\landTime) \leq y_{\mathrm{net}} \}. \label{landCond}
\end{aligned}
\end{equation}
%
The landing time $\landTime$ at which the ball hits the horizontal table plane, is found using the ball prediction functions
%
\begin{equation}
\begin{aligned}
\landTime(\joint(T),\dot{\joint}(T),T) = \{t > \netTime | \, b_z(t) \!=\! \ztable + \ballRadius\} \label{landTime}. 
\end{aligned}
\end{equation}
%
\noindent The landing function $\landFun$ that predicts the future ball position on the horizontal table plane,
%
\begin{align}
&\landFun\big(\joint(T),\dot{\joint}(T), T\big) = \begin{pmatrix}
\landTime \\ b_{x}(\landTime) \\ b_{y}(\landTime) \\ \ztable + \ballRadius
\end{pmatrix}, \label{switchCond3}
\end{align}
%
\noindent is, as before, the composition of a ball-racket contact model with the ball flight model.
%
\subsection{Nonlinear Constrained Optimization}\label{nco2}
%
We briefly show here that the solution $\joint(t)$ to the original optimal control problem posed in \mbox{\eqref{costFnc1} -- \eqref{initCond2}}, with additional penalties for landing and hitting, is a third order polynomial for each degree of freedom, $i = 1, \ldots, n \ $. The penalties for landing and hitting can be grouped together as $\phi_{\mathrm{pen}}$, where
%
\begin{align}
\phi_{\mathrm{pen}} &= \weightHit\phi_{\mathrm{hit}}(\joint_f, \hitTime) + \weightLand\phi_{\mathrm{land}}(\joint_f,\dot{\joint}_f, \hitTime),\\
\phi_{\mathrm{hit}} &= (\ballPred(T) - \racket(T))^{\mathrm{T}}\projRacket(T)(\ballPred(T) - \racket(T)), \\ 
\phi_{\mathrm{land}} &= (\ballPred(\landTime) - \ballLand)^{\mathrm{T}}(\ballPred(\landTime) - \ballLand).
\end{align}
%
\noindent with tunable weights $\weightHit$ and $\weightLand$.
%
\begin{figure*}
	\centering
	\def\svgwidth{1.3\columnwidth}
	\input{Figures/graphicalModel.pdf_tex}
	\caption{Graphical representation of table tennis interactions. The hybrid system for the table tennis ball is described by the flight dynamics, governed by a set of differential equations, as well as a discrete hitting event $\hit$ that changes the ball velocity from $\dot{\ballPred}(\hitTime^{-})$ to $\dot{\ballPred}(\hitTime^{+})$ at the hitting time $\hitTime$. The control variables for the reduced optimization problem are located in the light blue rectangle. Racket constraints that are enforced by $\Alg$ to land the ball to a fixed location are indicated in the red rectangle. $\AlgTwo$ on the other hand, directly enforces the task (landing and net) constraints, located in the orange rectangle, without additional constraints. By additionally checking for the hitting condition $\hit$ in the optimization, this problem can be cast as a (standard) continuous optimal control problem, where the decision variables $\joint_f$, $\dot{\joint}_f$ and $\hitTime$ continuously affect the outgoing ball velocity, the ball net and landing positions, through the repeated application of the flight model \eqref{flightModel} and the contact model \eqref{contactModel}.} 
	\label{graphTT}
\end{figure*}
%
\paragraph{\textbf{Derivation from Maximum Principle}.} The same derivation in section~\ref{nco1} applies for the Hamiltonian and the momenta. Instead of the boundary equality constraints we get the more general inequality constraints at striking time
%
\begin{align}
\hamiltonian(T) &= \frac{\partial\Phi}{\partial T}, \label{bound_trans1}\\
-\momentaPos^{*}(T) &= \frac{\partial\Phi}{\partial\joint(T)}, \label{bound_trans2}\\
-\momentaVel^{*}(T) &= \frac{\partial\Phi}{\partial\dot{\joint}(T)}, \label{bound_trans3}
\end{align}
%
\noindent where the generalized boundary cost is
%
\begin{align}
\Phi(\joint,\dot{\joint},\hitTime) &= \phi_{\mathrm{pen}} + \vec{\nu}^{\mathrm{T}}\vec{\Psi}_{\mathrm{strike}}
\end{align}
%
for some Lagrange multipliers $\vec{\nu} \in \mathbb{R}^{13}$ and $\vec{\Psi}_{\mathrm{strike}} \leq \vec{0}$ representing the hitting, landing and net inequality constraints $\mbox{\eqref{hitFunc} -- \eqref{landFunc}}$. The conditions $\mbox{\eqref{bound_trans1} -- \eqref{bound_trans3}}$ along with primal feasibility, complementary slackness and dual feasibility conditions
%
\begin{align}
&\vec{\Psi}_{\mathrm{strike}}(\joint(\hitTime),\dot{\joint}(\hitTime),\hitTime) \leq \vec{0}, \\
&\vec{\Psi}_{\mathrm{strike}}^{\mathrm{T}}\vec{\nu} = 0, \\
&\vec{\nu} \geq \vec{0}, 
\end{align}
%
respectively, supply the additional equations to determine all the variables. 

%\section*{Joint limits}
%
%The joint position, velocity and acceleration limits $\theta_{\mathrm{MIN}}$, $\dot{\theta}_{\mathrm{MAX}}$, $\ddot{\theta}_{\mathrm{MAX}}$ for the Barrett WAM are shown in Table~\ref{joint limits} where the shoulder joints are noted as SFE, SAA, HR, elbow joint as EB and the wrist joints as WR, WFE and WAA. The range of allowed velocities and accelerations is symmetric, i.e. $\dot{q}_i \in [-\dot{\theta}_{\mathrm{MAX}},\dot{\theta}_{\mathrm{MAX}}]$ and $\ddot{q}_i \in [-\ddot{\theta}_{\mathrm{MAX}},\ddot{\theta}_{\mathrm{MAX}}]$, $i = 1, \ldots, 7$.
%
%\begin{table}
%\renewcommand{\arraystretch}{1.3}
%\caption{Joint Limits}
%\label{joint limits}
%\centering
%\begin{tabular}{c|c|c|c|c}
%& \bfseries $\theta_{\mathrm{MAX}}$ & $\theta_{\mathrm{MIN}}$ & \bfseries $\dot{\theta}_{\mathrm{MAX}}$ & \bfseries $\ddot{\theta}_{\mathrm{MAX}}$ \\
%\hline
%SFE & $2.60$ & $-2.60$ & $200$ & $200$ \\
%\hline
%SAA & $2.00$ & $-2.00$ & $200$ & $200$ \\
%\hline
%HR & $2.80$ & $-2.80$ & $200$ & $200$ \\
%\hline
%EB & $3.10$ & $-0.90$ & $200$ & $200$ \\
%\hline
%WR & $1.30$ & $-4.80$ & $200$ & $200$ \\
%\hline
%WFE & $1.60$ & $-1.60$ & $200$ & $200$ \\
%\hline
%WAA & $2.20$ & $-2.20$ & $200$ & $200$
%\end{tabular}
%\end{table}
%
%
\paragraph{\textbf{Parameter Optimization}.} With the same parameterization as in $\Alg$ ($\alg$), the cost functional is the same with more general inequality constraints
%
\begin{align}
\min_{\joint_f,\dot{\joint}_f, \hitTime} & \,3\hitTime^3 \vec{a}_3^{\mathrm{T}}\vec{R}\vec{a}_3 + 3\hitTime^2 \vec{a}_3^{\mathrm{T}}\vec{R}\vec{a}_2 +
\hitTime\vec{a}_2^{\mathrm{T}}\vec{R}\vec{a}_2 + \phi_{\mathrm{pen}} \label{costFnc4} \\
\textrm{s.t. \ }
& \vec{\Psi}_{\mathrm{strike}}(\joint_f,\dot{\joint}_f,\hitTime) \leq \vec{0}, \\
& \jointMin \leq \joint_f \leq \jointMax, \\
& \jointMin \leq \joint_{\mathrm{ext}} \leq \jointMax,
\end{align}
%
\noindent where the components $\phi_{\mathrm{land}}(\joint_f,\dot{\joint}_f, \hitTime)$ and $\phi_{\mathrm{hit}}(\joint_f, \hitTime)$ of $\phi_{\mathrm{pen}}$, impose additional penalties on the hitting joint positions and velocities. Unlike the $\alg$, the joint extrema $\joint_{\mathrm{ext}}$ are only checked for the striking trajectory, as the returning trajectory parameters are the result of an additional optimization. 
%
\paragraph{\textbf{Resting State Optimization}.} For the $\algTwo$, we additionally consider a resting posture optimization to find a more \emph{defensive} posture for the robot. By finding a joint resting state $\joint_0$ that minimizes both the distance from the hitting state $\joint_f$ and the squared Frobenius norm of the Jacobian at the resting state
%
\begin{align}
\min_{\joint_0,t} & \, (\joint_0 - \joint_f)^{\mathrm{T}}(\joint_0 - \joint_f) + \|\jac(\joint_0)\|_{F}^{2} \label{resting-state-optim-cost} \\
\textrm{s.t. \ } 
& 0 \leq t \leq T_{\mathrm{pred}}, \\
& \kin_p(\joint_0) = \ballPred(t), \label{resting-state-optim-limfin} \\ 
& \jointMin \leq \joint_0 \leq \jointMax, \label{resting-state-optim-limtraj} \\
& \jointMin \leq \joint_{\mathrm{ext}} \leq \jointMax, \label{resting-state-optim-fin}
\end{align}
%
\noindent such that the Cartesian resting state intersects the ball path for some $t, \ 0 \leq t \leq T_{\mathrm{pred}}$, we can minimize the amount of movement necessary to return the next incoming ball. The feasibility of the third order polynomials that goes from hitting state $\joint_f, \, \dot{\joint}_f$ to $\joint_0, \, \dot{\joint}_0 = \vec{0}$ is ensured by including the joint extrema candidates throughout the returning trajectory in \eqref{resting-state-optim-fin}. Including the Frobenius norm of the Jacobian in the cost function makes sure that the striking trajectories will be easy to generate (i.e., have low accelerations) for the next predicted ball trajectories near the last ball trajectory.

\paragraph{\textbf{Online Trajectory Generation}.} The resulting table tennis player is summarized in pseudocode format in Algorithm~\ref{alg2}. As in Algorithm~\ref{alg1}, the algorithm can be run online whenever there are enough ball samples $\numBallsMin \approx \minball$ available to estimate the incoming ball reliably. After computing an initial striking trajectory and starting to move, the trajectories can be corrected online (line 11-13) whenever there are new ball samples available. Compared to $\Alg$, the gained flexibility due to relaxed constraints is increased with the addition of the resting posture optimization (line 12) that reduces the accelerations of the next hitting movements with similar incoming balls. %at the cost of a more involved optimization that is harder to solve.

\begin{algorithm}[t]
	\begin{mdframed}
		\small\sf%\centering
		\caption{$\AlgTwo$ ($\algTwo$)}
		\label{alg2}
		\begin{minipage}{\linewidth}
			\begin{algorithmic}[1]
				\Require $\joint_{0}, \vec{R}, \numBallsMin, \predTime, \weightHit, \weightLand$ %\State {\bfseries Input:} 
				%\STATE Estimate model parameters from demonstrations
				\State Wait at initial posture $\joint_{0}$.
				\Loop
				\State Query vision sys. for new observation $\ball_{\mathrm{obs}}$.
				\State Observe current state $\joint_{\mathrm{cur}}, \dot{\joint}_{\mathrm{cur}}$.
				\If{$\numBallsMin$ new ball observations $\ball_{\mathrm{obs}}$}
				\State Initialize EKF.
				\EndIf
				\If{EKF is initialized \algorithmicAnd valid obs. $\ball_{\mathrm{obs}}$}
				\State Estimate position $\ball$ and vel. $\dot{\ball}$ with EKF.
				\State Predict $\ballPred(t)$ till horizon $\predTime$.
				\State \parbox[t]{\dimexpr\linewidth-\algorithmicindent}{Compute $\joint_f, \dot{\joint}_f, T$ from $\joint_{\mathrm{cur}}, \dot{\joint}_{\mathrm{cur}}$ using $\ballPred(t)$ \\and the weights $\vec{R}, \weightHit, \weightLand$.}
				\State Compute $\joint_0$ using $\joint_f, \ballPred(t)$ %: $\mbox{\eqref{resting-state-optim-cost} -- \eqref{resting-state-optim-fin}}$
				\State \parbox[t]{\dimexpr\linewidth-\algorithmicindent}{Update strike and return trajectories \\$\joint_{\mathrm{des}}(t) = \{\joint_{\mathrm{strike}}(t), \joint_{\mathrm{return}}(t)\}$.}
				\EndIf
				\State \parbox[t]{\dimexpr\linewidth-\algorithmicindent}{Track $\joint_{\mathrm{des}}(t)$ with Inv. Dyn. $\vec{\tau} = \invdyn(\joint_{\mathrm{des}},\dot{\joint}_{\mathrm{des}},\ddot{\joint}_{\mathrm{des}})$.} % Execute
				\EndLoop
			\end{algorithmic}
		\end{minipage}
	\end{mdframed}
\end{algorithm}