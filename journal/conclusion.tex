\section{Conclusion}\label{end}

In this paper we have presented two new algorithms ($\alg$ and $\algTwo$) for generating table tennis striking trajectories that extend previous work in table tennis strike movement generation. 

\subsection{Summary of the Contributions}

The two table tennis players use an optimal-control based approach for generating striking trajectories. The striking and return trajectories are third order polynomials that intercept the ball at the optimized hitting point at the optimized hitting time. Unlike previous approaches, our optimization based framework respects the joint limits, while leading to efficient movements with low accelerations. Furthermore, by varying the hitting time $T$ the problem of finding feasible joint trajectories is simplified. Further constraints can be easily imposed on the system, and we have considered, for instance, racket constraints for $\alg$ and an additional resting posture optimization for $\algTwo$. 

The optimizations can be run online in the robotic setup shown in Figure~\ref{robot} and given new joint position and ball position measurements, the trajectories are updated. Correcting for new ball positions, by repeating optimization, makes our table tennis players more robust to execution errors and inaccuracies in ball estimation $\&$ prediction. We show the performance of our two table tennis players in the real robot platform and compare with previous approaches.

\subsection{Outlook \& Future Work}

The two players $\Alg$ and $\AlgTwo$ can generate trajectories more flexibly than before and lead to two different play-styles which could potentially be utilized by a higher-level strategy. We believe that this is a promising direction, where a higher level learning algorithm could switch between different trajectory generation schemes. The weights and the additional parameterization for the two algorithms can be explored based on feedback on the robot's performance. Reinforcement learning~\citep{Sutton98} with rewards based on observed ball landing positions, provides a suitable framework to tune the proposed algorithms' performance online.

Finally, the cost functionals that we have introduced consider the accelerations as the quantity to be minimized. Whenever the cancellation in feedback linearization is imperfect due to inaccurate robot dynamics models, execution errors will prevent the robot from achieving the desired trajectories or the minimal accelerations. A more robust and adaptive way to include execution errors in trajectory generation will be considered in future work.

% CRITICAL DISCUSSION: 
% too much model dependent